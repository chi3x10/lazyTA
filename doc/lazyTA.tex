\documentclass[11pt,letterpaper]{article}
\special{papersize=8.5in,11in}
\usepackage[left=1in,top=1in,right=1in,bottom=1in]{geometry}
\usepackage{enumerate}
\usepackage{graphicx}
\setlength{\parindent}{0in}
\setlength{\headsep}{8pt}
\setlength{\footskip}{5pt}
\parskip 0pt 



%opening
\title{LAZY TA}
\author{Jason Yu-Tseh Chi}



\begin{document}
%The exam will be inserted here. 

\section{Introduction}
LAZYTA is a software that 

\begin{itemize}\itemsep0pt
\item randomly choose questions from repositories of multiple choice questions. 
\item shuffles the order of the questions. 
\item shuffles the options of a question. 
\item generates multiple forms of the selected questions in latex format and answer keys to the forms. 
\item allows users to specify group questions. Group questions will remain together after the shuffling. 
\item allows users to specify questions that must be included. 
\end{itemize}

\section{Repository}

\subsection{A simple example}
Below is a sample exam repository. Just like html files, the basic structure of a repository file consists of tags. The $<$q$>$ and $<$/q$>$ tags define a question. Inside these two tags, the $<$b$>$ and $<$/b$>$ tags define the stem of this question. Each option of the question is enclosed by $<$c$>$ and $<$/c$>$ tags. The $<$b$>$ and $<$/b$>$ tags must always be placed before the $<$c$>$ and $<$/c$>$ tags. 

\begin{verbatim}
<q> 
<b>This is sample question one.</b>
<c> Choice 1.</c>
<c> Choice 2.</c>
<c> Choice 3.</c>
<c> Choice 4.</c>
<c> Choice 5.</c>
<!-- Comment -->
</q>
\end{verbatim}

The above question code will generate LaTex codes that can be "typeset-ed" into the following. Notice the order of the options is shuffled.

\begin{enumerate}[1.\underline{\makebox[0.5in]{    }}]
\setlength{\itemsep}{0pt}
\setlength{\parskip}{0pt}
\setlength{\parsep}{0pt}
\item This is sample question one.\\
    \begin{tabular}{l l l l l}
        \textsf{\textbf{(A)} Choice 1.}&
        \textsf{\textbf{(B)} Choice 3.}&
        \textsf{\textbf{(C)} Choice 2.}&
        \textsf{\textbf{(D)} Choice 5.}&
        \textsf{\textbf{(E)} Choice 4.}\\
    \end{tabular}\\
\end{enumerate}

\subsection{Question Properties}

\begin{description}
    \item[$<$ANS$>$] This property tag tells the program which option is the correct one. This is primarily used to generate the answer key files. When a question doesn't have an $<$ANS$>$ tag, a question mark will be placed in the answer key file. Also, in the TA form\footnote{Discussed in later section}, the correct answer will be marked with a box. This tag must be placed in between the $<$c$>$ and $<$/c$>$ tags. 

    \item[$<$NOA$>$] This tag creates a option "None of the above.". This option will always be the last option. This tag must be placed in between the $<$c$>$ and $<$/c$>$ tags. 

    \item[$<$noshuffle$>$] This tag disables the shuffling of options of a question. It must be placed between the $<$q$>$ and $<$/q$>$ tags and before the $<$b$>$ tag of the question. 

    \item[$<$layout=n$>$] Force the layout format. n is either 1,2,3, or 4. Without this property tag, the program automatically determinate layout style of a question based on the length of the longest option(\# of characters). This option allows users to set the layout of the question. Figure below demonstrates four different layout styles. 


\hspace{-0.65in}\rule{7in}{0.02in}
\begin{enumerate}[1.\underline{\makebox[0.5in]{    }}]
\setlength{\itemsep}{0pt}
\setlength{\parskip}{0pt}
\setlength{\parsep}{0pt}

%%Warning!!: There should not be any text between the \begin{enumerate.....
%%and the first \item{... 
%% You can copy the text or the \includegraphic lines above the \begin{enumerate}.... line
%% to prevent LaTEX compile error!! 


%% ----------------------------------------------------
\item An example of layout 1.\\
    \begin{tabular}{l l l l l}
        \textsf{\textbf{(A)} Option 2.}&
        \textsf{\textbf{(B)} Option 5.}&
        \textsf{\textbf{(C)} Option 4.}&
        \textsf{\textbf{(D)} Option 1.}&
        \textsf{\textbf{(E)} Option 3.}\\
    \end{tabular}\\
%% ----------------------------------------------------
\item An example of layout 2.\\
    \begin{tabular}{l l l}
        \textsf{\textbf{(A)} Option 5.}&
        \textsf{\textbf{(B)} Option 3.}&
        \textsf{\textbf{(C)} Option 1.}\\
        \textsf{\textbf{(D)} Option 2.}&
        \textsf{\textbf{(E)} Option 4.}&
\\
    \end{tabular}\\
%% ----------------------------------------------------
\item An example of layout 3.\\
    \begin{tabular}{l l}
        \textsf{\textbf{(A)} Option 5.}&
        \textsf{\textbf{(B)} Option 4.}\\
        \textsf{\textbf{(C)} Option 2.}&
        \textsf{\textbf{(D)} Option 1.}\\
        \textsf{\textbf{(E)} Option 3.}&
\\
    \end{tabular}\\
%% ----------------------------------------------------
\item An example of layout 4.\\
    \begin{tabular}{p{6in}}
        \textsf{\textbf{(A)} Option 2.}\\
        \textsf{\textbf{(B)} Option 4.}\\
        \textsf{\textbf{(C)} Option 1.}\\
        \textsf{\textbf{(D)} Option 3.}\\
        \textsf{\textbf{(E)} Option 5.}\\
    \end{tabular}\\

\hspace{-1in} \rule{7in}{0.02in}
%%===========================================================
\end{enumerate}

    \item[$<$keepme$>$] The question that contains this tag is guaranteed to be selected. It must be placed between the $<$q$>$ and $<$/q$>$ tags and before the $<$b$>$ tag of the question. 
\end{description}

Now, let's take a look at a more complicated example. 
\begin{verbatim}
<q><keepme><layout=4>
<b>Which state is nicknamed sunshine state?</b>
<c> New York.</c>
<c> Florida.<ANS></c>
<c> <NOA> </c> <!-- None of the above tag -->
<c> Alaska. </c>
<c> Texas.</c>
</q>
\end{verbatim}

The latex code generated from the above repository is typeseted into 
\begin{enumerate}[1.\underline{\makebox[0.5in]{    }}]
\setlength{\itemsep}{0pt}
\setlength{\parskip}{0pt}
\setlength{\parsep}{0pt}
\item Which state is nicknamed the sunshine state?\\
    \begin{tabular}{l l l l l}
        \textsf{\textbf{(A)} Texas.}&
        \textsf{\textbf{(B)} Alaska.}&
        \textsf{\textbf{(C)} Florida.}&
        \textsf{\textbf{(D)} New York.}&
        \textsf{\textbf{(E)} None of the above. }\\
    \end{tabular}\\
\end{enumerate}

\subsection{Group}
Group tag ($<$group$>$ and $<$/group$>$) are used when multiple questions must be placed together. Questions enlosed by $<$group$>$ and $<$/group$>$ can still be shuffled but only shuffled within the group. Below is an example of using group tags. 

\begin{verbatim}
<group><noshuffle>

<q><keepme><layout=4>
<b>Which state is nicknamed the sunshine state?</b>
<c> New York.</c>
<c> Florida.<ANS></c>
<c> <NOA> </c> <!-- None of the above tag -->
<c> Alaska. </c>
<c> Teax.</c>    </q>

<q> <b>Continue from previous question. What is the best university of the sunshine state?</b>
<c><NOA></c>
<c>University of Florida. <ANS></c>
<c>NYU.</c>
<c>Teax A\&M. </c>
<c>Collage of survial training. </c> </q>

</group>
\end{verbatim}
In the above example, it is very obvious that the 2nd question depends on the first one. Therefore it's not appropriate to shuffle the questions within group. The $<$noshuffle$>$ tag right after the $<$group$>$ tag tells the program not to shuffle the questions within this group. Note: the options of each question will still be shuffled unless a $<$noshuffle$>$ tag is inserted inside the question(between $<$q$>$ and $<$/q$>$  ). 
\subsection{Special tags}
\begin{description}
    \item[Comments] Just like html, comments must be placed in between '$<$!{-}{-}' and '{-} {-}$>$'. You can find examples of comments in the group questions example above. 

    \item[$<$img$>$ $<$/img$>$] This allows users to insert images in the exams. You must put the latex code that actually inserts the image. Here is an example: 
\begin{verbatim}
<group>
    <img>\includegraphics[bb=0 0 345 423, width=1.4in]{imgs/exam3_fig1.jpg}</img>
    <q><b>What is the latin name of the animal in the picture above?</b>
        <c>Option1</c><c>Option2</c><c>Option3</c><c>Option4</c><c><NOA></c>
    </q>
    <q><b>How would you eat it if it's actually legal to eat it?</b>
        <c>Option1</c><c>Option2</c><c>Option3</c><c>Option4</c><c><NOA></c>
    </q>
</group>
\end{verbatim}
An image can be inserted either in a group or a question. When it's inserted in a group, the $<$img$>$ must be placed right after the $<$group$>$ tag. The image will be inserted before the first question of the group. When it's inserted in a question, the $<$img$>$ tag must be placed right after the $<$q$>$ tag. The image will be inserted right before the question.

 
\end{description}

\section{Runnig the program}
Run make to compile the program. A binary file named lazyTA will be generated. To see the help of lazyTA, simply type
\begin{verbatim}
    ./lazyTA
\end{verbatim}
Here are some examples:
\begin{enumerate}
    \item Read repository file exam1.rep. A non-shuffled form (form 0) and a shuffled form (form A) along with their corresponding TA forms will be generated. Since the output filename is not specified, the output filenames will be EXAM\_Form\_XXX.tex
    \begin{verbatim}
./lazyTA -i exam1.rep
    \end{verbatim}

    \item Read only 40 questions from repository file exam1.rep. 
    \begin{verbatim}
./lazyTA -i exam1.rep 40
    \end{verbatim}

    \item Read a total of 50 questions from repository file exam1\_.rep and exam1\_2.rep. 
    \begin{verbatim}
./lazyTA -i exam1_1.rep -i exam1_2.rep -nq 50
    \end{verbatim}Numbers of questions selected from each repository will be determined proportionally based on the total number of questions in each repository. For example, there are 100 questions in exam1\_1.rep and 200 questions in exam1\_2.rep. $\frac{100}{(100+200)} \times 50$ questions will be selected from exam1\_1.rep and $\frac{200}{(100+200)} \times 50$ questions will be selected from exam1\_2.rep. 

    \item Read 10 questions from exam1\_.rep and 25 questions from exam1\_2.rep. 
    \begin{verbatim}
./lazyTA -i exam1_1.rep 10 -i exam1_2.rep 25
    \end{verbatim}

    \item Generate 3 different shuffled forms. The filenames will be EXAM\_Form\_A/B/C...tex.
    \begin{verbatim}
./lazyTA -i exam1_1.rep -nf 3
    \end{verbatim}

    \item Read from exam1.rep and the output filename will be midterm\_Form\_A.tex. 
    \begin{verbatim}
./lazyTA -i exam1_1.rep -o midterm
    \end{verbatim}

\end{enumerate}


\end{document}

